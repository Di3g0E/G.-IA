\documentclass{article}
\usepackage[utf8]{inputenc}
\usepackage{amsmath}
\usepackage{amsfonts}
\usepackage{amssymb}
\usepackage{graphicx}
\usepackage{listings}
\usepackage{color}
\usepackage{tcolorbox}
\usepackage[left=2cm,right=2cm,top=3.5cm,bottom=3.5cm]{geometry}

% Configuración de resaltado de sintaxis para código
\definecolor{codegreen}{rgb}{0,0.6,0}
\definecolor{codegray}{rgb}{0.5,0.5,0.5}
\definecolor{codepurple}{rgb}{0.58,0,0.82}
\definecolor{backcolour}{rgb}{0.95,0.95,0.92}

\lstdefinestyle{mystyle}{
    backgroundcolor=\color{backcolour},   
    commentstyle=\color{codegreen},
    keywordstyle=\color{magenta},
    numberstyle=\tiny\color{codegray},
    stringstyle=\color{codepurple},
    basicstyle=\ttfamily\footnotesize,
    breakatwhitespace=false,         
    breaklines=true,                 
    captionpos=b,                    
    keepspaces=true,                 
    numbers=left,                    
    numbersep=5pt,                  
    showspaces=false,                
    showstringspaces=false,
    showtabs=false,                  
    tabsize=2
}

\lstset{style=mystyle}

\title{Guía Práctica de Configuración de Servidor Linux}
\author{Tú Nombre}
\date{\today}

\begin{document}
\maketitle

\section{Introducción}
En esta guía, se detallan los pasos para configurar un servidor Linux desde cero, asegurarlo y mantenerlo operativo con una API-Rest en Python.

\subsection{Objetivos de la Práctica}
\begin{itemize}
  \item Configurar un servidor Linux accesible.
  \item Securizar el servidor y programar un script de mantenimiento.
  \item Instalar y mantener una API-Rest en Python.
\end{itemize}

\section{Paso a Paso}
\subsection{1. Administración Básica del Servidor}
Para comenzar, es necesario crear una instancia de EC2 en AWS (Amazon Elastic Compute Cloud) y conectarse a ella utilizando SSH.

\subsubsection{Conexión SSH}
Los siguientes pasos detallan cómo conectarse a la instancia mediante SSH desde Visual Studio Code:

\begin{enumerate}
  \item Conectar a la máquina virtual desde Visual Studio Code.
  \item Moverse a la carpeta de la clave de acceso al servidor.
  \item Ejecutar el comando: \texttt{chmod 400 "Nombre\_de\_la\_clave.pem"} en la terminal.
  \item Conectar mediante SSH usando: \\
  \texttt{ssh -i "Llave-Mi\_servidor\_web.pem" ubuntu@ec2-13-53-136-17.eu-north-1.compute.amazonaws.com}
\end{enumerate}

\subsection{2. Puesta en Marcha del Servidor Linux Accesible}
Para asegurar un servidor seguro, se deben elegir una imagen de sistema operativo Linux compatible al crear la instancia EC2 y configurar reglas de seguridad en el grupo de seguridad correspondiente.

\subsubsection{Cerrar Puertos de Entrada y Salida en Desuso}
Para ello, identifica el ID del Grupo de Seguridad y revoca el acceso a todos los puertos excepto el 22.

\begin{quote}
\texttt{curl -s http://169.254.169.254/latest/meta-data/security-groups/} \\
\texttt{aws ec2 revoke-security-group-ingress --group-id tu-id-de-grupo --protocol all --cidr 0.0.0.0/0}
\end{quote}

\begin{figure}[ht]
  \centering
  \includegraphics[width=0.7\linewidth]{Puertos_de_entrada_1.png}
  \caption{Configuración de Puertos de Entrada}
\end{figure}

\subsection{3. Securización y Programación de Script de Mantenimiento}
Se detallan los pasos para implementar Fail2ban, un sistema de prevención de intrusiones, y se muestra un script de mantenimiento para respaldar y limpiar archivos antiguos.

\subsubsection{Pasos para Implementar Fail2ban}
\begin{enumerate}
  \item Descargar e instalar Python (versión mayor o igual a 3.5).
  \item Actualizar el sistema y instalar Python y Pip.
  \item Instalar Fail2ban y configurarlo según los requisitos.
\end{enumerate}

\subsubsection{Script de Mantenimiento}
Se presenta un script Bash para realizar respaldos y limpiar archivos antiguos automáticamente.

\begin{lstlisting}[language=bash, caption=Script de Mantenimiento]
#!/bin/bash

directorios_respaldar="/etc /var/lib /home/*"
directorio_backup="/home/directorio_backup"

fecha=$(date +"%Y-%m-%d")
tar_file="$directorio_backup/backup_$fecha.tar.gz"
tar -czvf "$tar_file" $directorios_respaldar

find /ruta/a/logs -type f -name "*.log" -mtime +7 -exec rm {} \;

echo "Copia de seguridad realizada en $tar_file"
echo "Archivos de log más antiguos de 7 días eliminados."
\end{lstlisting}

\subsubsection{Automatización del Script}
Se muestra cómo automatizar la ejecución del script utilizando cron.

\begin{verbatim}
0 3 * * * /home/scripts/script_backup.sh > /home/directorio_backup/log_backup.log 2>&1
\end{verbatim}

\subsection{4. Servicios}
Se finaliza mostrando cómo crear una API básica con Bottle en Python para ofrecer servicios en el servidor.

\subsubsection{Crear la API Mínima con Bottle}
\begin{enumerate}
  \item Instalar el paquete Bottle.
  \item Implementar la lógica de los recursos requeridos en un archivo Python.
\end{enumerate}

\begin{lstlisting}[language=Python, caption=API Básica con Bottle]
from bottle import Bottle, run
import datetime

app = Bottle()

@app.route('/hi')
def hello():
    now = datetime.datetime.now()
    return f'Hola, hoy es {now.strftime("%d/%m/%Y")} y son las {now.strftime("%H:%M:%S")}'

@app.route('/status')
def status():
    # Implementar lógica para listar servicios en ejecución
    return "Servicios en ejecución: Servicio1, Servicio2, ..."

if __name__ == '__main__':
    run(app, host='0.0.0.0', port=8080)
\end{lstlisting}

\end{document}
