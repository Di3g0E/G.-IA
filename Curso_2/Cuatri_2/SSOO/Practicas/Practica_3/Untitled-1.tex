\documentclass{article}
\usepackage[utf8]{inputenc}
\usepackage{amsmath}
\usepackage{amsfonts}
\usepackage{amssymb}
\usepackage{listings} % Para resaltar la sintaxis del código
\usepackage{graphicx}
\usepackage{tcolorbox}
\usepackage{color}
\usepackage{hyperref}

\usepackage[left=2cm,right=2cm,top=3.5cm,bottom=3.5cm]{geometry}

\definecolor{codegreen}{rgb}{0,0.6,0}
\definecolor{codegray}{rgb}{0.5,0.5,0.5}
\definecolor{codepurple}{rgb}{0.58,0,0.82}
\definecolor{backcolour}{rgb}{0.95,0.95,0.92}

\lstdefinestyle{mystyle}{
    backgroundcolor=\color{backcolour},   
    commentstyle=\color{codegreen},
    keywordstyle=\color{magenta},
    numberstyle=\tiny\color{codegray},
    stringstyle=\color{codepurple},
    basicstyle=\ttfamily\footnotesize,
    breakatwhitespace=false,         
    breaklines=true,                 
    captionpos=b,                    
    keepspaces=true,                 
    numbers=left,                    
    numbersep=5pt,                  
    showspaces=false,                
    showstringspaces=false,
    showtabs=false,                  
    tabsize=2
}

\lstset{style=mystyle}


% Portada
\title{\Huge \textbf{Práctica 3. Introducción a SSOO de Tiempo Real}}
\author{Diego Esclarín Fernández}
\date{\today}


\begin{document}

\begin{titlepage}
    \maketitle
    \ 
    \begin{figure}[h]
        \centering
        \includegraphics[width=0.6\textwidth]{Logo_URJC.png}
        \label{fig:ejemplo}
    \end{figure}
\end{titlepage}


% Contenido del documento
\section{Sistemas Operativos de Tiempo Real}

\subsection{ ¿Qué son los sistemas operativos de tiempo real?}
Los sistemas operativos de tiempo real (RTOS) son usados en el mundo de la computación donde el tiempo es
un recurso crítico. Estos sistemas se han diseñaado para procesar datos y ejecutar tareas en tiempo real, ofreciendo
respuestas rápidas y predecibles, lo cual es necesario para controlar procesos industriales, vehículos autónomos,
sistemas médicos de emergencia, entre otros. A diferencia de los sistemas operativos de propósito general, que están
diseñados para una amplia gama de tareas de procesamiento, los RTOS están optimizados para garantizar que las
tareas críticas se completen en el tiempo requerido. Esto se logra a través de la gestión eficiente de las prioridades
de las tareas y el uso de algoritmos de planificación que pueden manejar un alto grado de interrupciones y cambios
de contexto, asegurando así la estabilidad y la fiabilidad del sistema \cite{ref1} \cite{ref2} \cite{ref5} \cite{ref6}.

\subsection{ Características principales de los sistemas operativos de tiempo real}
\begin{itemize}
  \item \textbf{Determinismo}: Realizan operaciones en instantes de tiempo fijos predeterminados o dentro de intervalos de
  tiempo predeterminados \cite{ref1}.

  \item \textbf{Uso eficiente de la memoria}: Los sistemas operativos de tiempo real (RTOS) están diseñados para ser
  ligeros y eficientes. Están opttimizados para aplicaciones integradas que normalmente tienen recursos limitados,
  incluyendo la memoria. Por lo tanto, los RTOS están diseñados para utilizar la menor cantidad de memoria
  posible, lo que permite que se ejecuten en dispositivos con recursos limitados \cite{ref2}.
  
  \item \textbf{Respuesta rápida a eventos}: Los RTOS están diseñados para responder rápidamente a los eventos. Esto se
  debe a que muchas aplicaciones de tiempo real tienen requisitos estrictos de tiempo de respuesta. Por ejemplo,
  en un sistema de control de tráfico aéreo, un retraso en la respuesta a un evento podría tener consecuencias
  desastrosas. Por lo tanto, los RTOS están diseñados para garantizar que puedan responder a los eventos dentro
  de un límite de tiempo predecible \cite{ref2} \cite{ref6}.

  \item \textbf{Multi-arquitectura}: El código puede ser portado a cualquier tipo de CPU \cite{ref1} \cite{ref2}.
  
  \item \textbf{Planificación de procesos en tiempo real}: La planificación de procesos en tiempo real es una característica
  clave de los RTOS. Los RTOS utilizan algoritmos de planificación especializados para determinar qué tarea
  debe ejecutarse en un momento dado. Estos algoritmos toman en cuenta las prioridades de las tareas y sus
  requisitos de tiempo para asegurar que todas las tareas se ejecuten de manera oportuna \cite{ref1} \cite{ref2}.
  
  \item \textbf{Comunicación interprocesos}: Las comunicaciones entre procesos están soportadas de una manera precisa
  y estable.
  
  \item \textbf{Control por parte del usuario}: Los usuarios tienen un mayor control sobre los procesos en comparación con
  los sistemas operativos de propósito general. Esto se debe a que los RTOS están diseñados para permitir a los
  usuarios controlar de manera precisa cómo y cuándo se ejecutan las tareas. Esto es especialmente importante
  en aplicaciones de tiempo real, donde el tiempo de ejecución de las tareas puede ser crítico \cite{ref1} \cite{ref2} 
  \cite{ref5} \cite{ref6} \cite{ref7} \cite{ref10} \cite{ref11}.
\end{itemize}


\newpage

\subsection{Diferencias con otros tipos de sistemas operativos}

Los sistemas operativos de tiempo compartido y los sistemas operativos en tiempo real (RTOS) representan dos
enfoques distintos para la gestión de tareas en la informática. Los sistemas operativos de tiempo compartido están
diseñados para permitir que múltiples usuarios utilicen un sistema informático de manera interactiva y simultánea.
Esto se logra mediante un proceso de planificación sofisticado que asigna tiempo de CPU a cada usuario de manera
eficiente, dando la apariencia de que todos los programas se ejecutan al mismo tiempo. Por otro lado, los RTOS
están optimizados para aplicaciones críticas donde el tiempo de respuesta es crucial. Aunque están diseñados para
enfocarse en una tarea principal a la vez, los RTOS modernos son capaces de multitarea gracias al cambio de
contexto, un mecanismo que les permite cambiar entre tareas de manera rápida y controlada. Este proceso es
fundamental para mantener la operación fluida y responder a eventos en tiempo real sin comprometer la fiabilidad
del sistema \cite{ref4} \cite{ref9}.

El cambio de contexto en los RTOS es un procedimiento altamente optimizado que minimiza el tiempo de
inactividad de la CPU durante las transiciones entre tareas. Esto es esencial en sistemas embebidos, como los
utilizados en dispositivos médicos, sistemas de control industrial y automóviles, donde el fallo o la demora en la
ejecución de una tarea puede tener consecuencias graves. Los RTOS aseguran que las tareas críticas reciban la
atención de la CPU cuando lo necesiten, priorizando las operaciones según su importancia y urgencia. Además, los
RTOS a menudo emplean técnicas como el “preemption”, que permite interrumpir una tarea de menor prioridad
para dar paso a una más crítica, garantizando así que los tiempos de respuesta sean siempre los adecuados.
El “preemption” en un Sistema Operativo de Tiempo Real (RTOS) se refiere a la capacidad del sistema operativo
de interrumpir una tarea en ejecución para dar paso a otra tarea de mayor prioridad \cite{ref2} \cite{ref5} \cite{ref6}.


\subsection{Casos de uso de los sistemas operativos de tiempo real}

Los sistemas operativos de tiempo real se utilizan en aquellos dispositivos o sistemas donde el tiempo de respuesta
es crítico. Estos sistemas se utilizan comúnmente en aplicaciones de control de procesos industriales, sistemas de
navegación automotriz, sistemas médicos y aeroespaciales, entre otros. Un ejemplo común de uso de los RTOS es en
los sistemas de control de tráfico aéreo. Estos sistemas requieren una respuesta en tiempo real a los eventos, ya que
cualquier retraso podría tener consecuencias graves. Los RTOS permiten que estos sistemas respondan rápidamente
a los eventos y manejen múltiples tareas simultáneamente, como el seguimiento de múltiples aviones al mismo
tiempo \cite{ref2} \cite{ref8}.


\subsection{Cómo se utilizan los sistemas operativos de tiempo real}

Los sistemas operativos de tiempo real se utilizan para facilitar la multitarea y la integración de tareas en diseños
con recursos y tiempo limitados. Se utilizan en una amplia variedad de aplicaciones donde la velocidad y la precisión
son críticas. Algunos ejemplos comunes de uso incluyen sistemas de control de tráfico aéreo, sistemas de navegación
GPS, sistemas de control industrial, entre otros  \cite{ref2} \cite{ref5} \cite{ref6} \cite{ref8}.


\newpage

\section{Implementación de los sistemas operativos de tiempo real}

\subsection{Desarrollo y depuración de aplicaciones en RTOS}
El desarrollo de aplicaciones en RTOS implica la creación de tareas o procesos que se ejecutan en bucles infinitos, cada 
uno responsable de una función. Estas tareas se ejecutan de forma independiente en su propio tiempo (aislamiento temporal) 
y pila de memoria (aislamiento espacial). Los programadores en el RTOS controlan qué tarea debe ejecutarse en la CPU y existen 
diferentes algoritmos de programación \cite{ref1}.

\subsection{Consideraciones de seguridad y fiabilidad}
Los sistemas operativos de tiempo real son un tipo concreto de sistema desarrollado para ejecutar aplicaciones que disponen 
de algún tipo de restricción temporal. Estos sistemas están diseñados para responder de manera adecuada ante varias formas de 
fallo, incluso concurrentes \cite{ref10}. En aplicaciones críticas para la seguridad, como dispositivos médicos y sistemas automotrices, 
cualquier falla en el sistema operativo podría tener consecuencias catastróficas. Como resultado, RTOS debe probarse y certificarse 
rigurosamente para cumplir con los estándares de seguridad y confiabilidad de la industria \cite{ref12}.


\section{Tendencias y desarrollos en sistemas operativos de tiempo real}

\subsection{Incorporación de técnicas de inteligencia artificial para mejorar la planificación de tareas}
La inteligencia artificial (IA) contribuye a mejorar la planificación de tareas mediante la automatización y asignación inteligente 
de las mismas. Esto se logra analizando la información disponible sobre las habilidades y capacidades de los miembros del equipo, así 
como la carga de trabajo actual y los plazos de entrega \cite{ref13}.

\subsection{Soporte para sistemas ciberfísicos y el Internet de las cosas (IoT)}
Un sistema ciberfísico se puede entender como la integración total del mundo material con un mundo paralelo implementado en software. 
Esto permite la comunicación tanto de objetos como de personas y/o servicios entre sí permitiendo optimizar procesos tanto cotidianos 
como comerciales \cite{ref14}. El Internet de las cosas (IoT) sirve para conectar y automatizar objetos físicos cotidianos a través de Internet, 
permitiendo la recolección, análisis y uso de datos en tiempo real \cite{ref15}.

\subsection{Avances en la seguridad y la certificación de sistemas críticos}
Los sistemas operativos de tiempo real deben probarse y certificarse rigurosamente para cumplir con los estándares de seguridad y 
confiabilidad de la industria \cite{ref12}. Recientemente, se ha publicado una nueva guía CCN-STIC de perfilado de seguridad para Windows y se han 
actualizado otras tres guías CCN-STIC sobre el procedimiento de acreditación de entidades implementadoras de guías CCN-STIC \cite{ref3}.



\newpage


\begin{thebibliography}{99} % Usamos 99 si esperamos menos de 100 referencias
    
    \bibitem{ref1} Wikipedia. \textit{Sistema operativo de tiempo real - Wikipedia, la enciclopedia libre}. Disponible en: \url{https://es.wikipedia.org/wiki/Sistema_operativo_de_tiempo_real}

    \bibitem{ref2} Digikey. \textit{Sistemas operativos en tiempo real (RTOS) y sus aplicaciones}. Disponible en: \url{https://www.digikey.com/es/articles/real-time-operating-systems-and-their-applications}
    
    \bibitem{ref3} CN-CERT. \textit{El CCN publica una nueva guía CCN-STIC de perfilado de seguridad para Windows y actualiza otras tres}. Disponible en: \url{https://www.ccn-cert.cni.es/es/seguridad-al-dia/novedades-ccn-cert/12948-el-ccn-publica-una-nueva-guia-ccn-stic-de-perfilado-de-seguridad-para-windows-y-actualiza-otras-tres.html}
    
    \bibitem{ref4} Gadget Info. \textit{Diferencia entre el tiempo compartido y el sistema operativo en tiempo real}. Disponible en: \url{https://es.gadget-info.com/difference-between-time-sharing}
    
    \bibitem{ref5} Herschel González. \textit{¿Qué es un Sistema Operativo en Tiempo Real y para qué sirve?}. Disponible en: \url{https://herschelgonzalez.com/que-es-un-sistema-operativo-en-tiempo-real-y-para-que-sirve/}
    
    \bibitem{ref6} Paola Coyla. \textit{¿Qué es un Sistema Operativo en Tiempo Real?}. Disponible en: \url{https://electronicaonline.net/software-y-apps/aplicaciones/sistema-operativo-en-tiempo-real-rtos/}
    
    \bibitem{ref7} EcuRed. \textit{Sistema Operativo de Tiempo Real - EcuRed}. Disponible en: \url{https://www.ecured.cu/Sistema_Operativo_de_Tiempo_Real}
    
    \bibitem{ref8} TecnoFAQ. \textit{¿Dónde Se Utilizan Los Sistemas Operativos En Tiempo Real? - FAQs De Tecnología}. Disponible en: \url{https://tecnofaq.com/donde-se-utilizan-los-sistemas-operativos-en-tiempo-real/}
    
    \bibitem{ref9} Comprendamos. \textit{Comparativa de sistemas operativos de tiempo real y sistemas operativos de tiempo compartido, para plataforma intel x86 y sus compatibles}. Disponible en: \url{https://www.comprendamos.org/alephzero/63/comparativadesi.html}
    
    \bibitem{ref10} INCIBE-CERT. \textit{Sistemas operativos de tiempo real, bastionado y funcionamiento}. Disponible en: \url{https://www.incibe.es/incibe-cert/blog/sistemas-operativos-tiempo-real-bastionado-y-funcionamiento}
    
    \bibitem{ref11} TIPOSDE. \textit{Tipos De Sistemas Operativos Tiempo Real}. Disponible en: \url{https://tiposde.net/tipos-de-sistemas-operativos-tiempo-real/}

    \bibitem{ref12} CronicaWeb. \textit{Sistemas operativos en tiempo real: aplicaciones y desafíos}. Disponible en: \url{https://www.cronicaweb.com/sistemas-operativos-en-tiempo-real-aplicaciones-y-desafios/}

    \bibitem{ref13} Appian. \textit{Optimización de procesos mediante Inteligencia Artificial: 5 formas de potenciar la eficiencia empresarial}. Disponible en: \url{https://appian.com/es/blog/acp/process-automation/ai-process-optimization-how-use.html}

    \bibitem{ref14} CodigoIoT. \textit{Internet de las Cosas}. Disponible en: \url{https://www.codigoiot.com/internet-de-las-cosas/}

    \bibitem{ref15} Digixem360. \textit{Internet de las cosas (IoT): ¿qué es, cómo funciona y ejemplos?}. Disponible en: \url{https://impactotic.co/tecnologia/internet-de-las-cosas-iot/}

\end{thebibliography}

\end{document}
